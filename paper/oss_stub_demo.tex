\documentclass{article}
\usepackage{amsmath}
\usepackage{graphicx}
\title{Demonstration of the OSS Stub Backend}
\author{SciResearch-AI}
\date{\today}
\begin{document}
\maketitle

\begin{abstract}
We present a synthetic research article generated entirely by the OSS stub
backend included with SciResearch-AI. The goal is to illustrate the LaTeX
output format without requiring the full 120B parameter model.
\end{abstract}

\section{Introduction}
Large language models are capable of drafting technical papers when paired with
structured reasoning loops. The OSS stub backend reproduces this process with
deterministic placeholder text, making it ideal for offline testing.

\section{Methodology}
Our pipeline follows a plan--write--revise loop. At each step the model produces
candidate paragraphs, which are then critiqued and refined. The stub backend
short-circuits generation and emits predefined sentences so that the rest of
the system can be exercised without heavy dependencies.

\section{Results}
Table~\ref{tab:ablations} summarizes the quality of the stub output across
several hypothetical metrics. Although the numbers are fictitious, the table
demonstrates that the tool chain correctly handles typical LaTeX constructs.

\begin{table}[h]
  \centering
  \begin{tabular}{lcc}
    \hline
    Metric & Baseline & Stub \\
    \hline
    Coherence score & 0.00 & 0.99 \\
    Citation accuracy & 0 & 42 \\
    \hline
  \end{tabular}
  \caption{Imaginary ablation study produced by the stub backend.}
  \label{tab:ablations}
\end{table}

\section{Conclusion}
This document serves as a high-quality placeholder that mimics the structure of
an authentic research paper. When the full OSS model weights are available, the
same pipeline can produce genuine content.

\bibliographystyle{plain}
\begin{thebibliography}{9}
\bibitem{demo}
A.~Author.
\newblock {Placeholder References for Demonstration}.
\newblock \emph{Journal of Synthetic Results}, 42(1):1--2, 2024.
\end{thebibliography}

\end{document}
