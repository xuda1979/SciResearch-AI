\documentclass{article}
\usepackage[utf8]{inputenc}
\title{Proof-Carrying Answers and Property-Based Self-Fuzzing}
\author{Research Team}
\date{\today}

\begin{document}
\maketitle

\begin{abstract}
Proof-Carrying Answers (PCA) require each model output to include a
machine-checkable proof or verification script. Property-based
self-fuzzing complements PCA by stress-testing claims with generated
counterexamples. Together they increase the reliability and
verifiability of large language model responses.
\end{abstract}

\section{Introduction}
Modern large language models often emit unverifiable statements.
Packaging every answer with proof artifacts helps mitigate
hallucination, while self-fuzzing probes weaknesses by searching for
counterexamples.

\section{Method}
For each claim, the system emits executable code that checks the claim
and attempts to find counterexamples. The code runs in a sandboxed
environment to ensure safety.

\section{Implementation}
The file \texttt{code.py} demonstrates how to generate a
proof-carrying answer using the \texttt{sciresearch\_ai} module and how
to fuzz a simple property.

\section{Conclusion}
Combining PCA with property-based self-fuzzing provides a lightweight
approach to increase trust in language model outputs by demanding
explicit proofs and automatically searching for violations.

\end{document}
