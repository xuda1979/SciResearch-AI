\documentclass{article}
\usepackage{hyperref}
\title{Counterparty Models with Verifiable Non-Collusion}
\author{SciResearch AI Team}
\date{\today}
\begin{document}
\maketitle
\begin{abstract}
\textbf{Counterparty Models with Verifiable Non-Collusion (CNCM)} reduce
hallucinations and bias by running specialised models as adversarial
partners: an Advocate, a Skeptic, and a Judge. Each model produces,
challenges, and aggregates answers while cryptographic transcripts
prevent collusion.
\end{abstract}
\section{Introduction}
Large language models often lack self-critical mechanisms. CNCM proposes
a multi-agent framework where an Advocate offers answers, a Skeptic
produces falsifiers, and a Judge aggregates under proper scoring rules.
The agents are isolated and exchange signed transcripts to discourage
collusion.
\section{Method}
The Advocate aims to maximise the utility of correct answers. The
Skeptic adversarially seeks counterexamples. The Judge rewards or
penalises each agent based on correctness using proper scoring rules.
Non-collusion is enforced by running agents on different architectures
or data slices and hashing their transcripts.
\section{Implementation}
A prototype pairs different models or fine-tuned variants. The Advocate
and Skeptic generate claims and counterclaims stored as signed
transcripts. The Judge, implemented as a separate model or algorithm,
resolves conflicts and produces a final answer.
\section{Conclusion}
CNCM harnesses model diversity and ensures that answers are challenged
and verified before being delivered.
\end{document}
