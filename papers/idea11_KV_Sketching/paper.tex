\documentclass{article}
\usepackage{amsmath}
\usepackage{hyperref}
\usepackage{listings}
\usepackage{graphicx}

\title{Idea 11: KV Sketching for Long Contexts}
\author{SciResearch-AI}
\date{\today}

\begin{document}
\maketitle

\begin{abstract}
Large language models (LLMs) equipped with extremely long context
windows suffer from high memory bandwidth requirements and prohibitive
VRAM usage because key and value (K/V) states must be stored for every
token.  ``KV-Sketching'' is a family of techniques that learns to
compress K/V state tensors into compact sketches that can be
rehydrated on demand.  By combining low-rank approximations,
count-sketch hashing and quantization, one can achieve multi-million-token
contexts without a linear growth in memory footprint.
\end{abstract}

\section{Problem}
Attention mechanisms require storing a matrix of key and value vectors for
every layer and token.  For a context window of one million tokens the
uncompressed K/V state can exceed tens of gigabytes, making inference on
consumer hardware impossible.  Existing sparse attention kernels (e.g.,
FlashAttention-3) reduce compute but not the memory footprint.

\section{Method}
KV-Sketching trains lightweight adapters on top of a frozen LLM that
produce sketchable K/V states.  Each layer yields two outputs:
\begin{enumerate}
  \item A low-dimensional projection of the K/V state (learned low-rank
        basis and quantized coefficients).
  \item An uncertainty score that determines whether the sketch is
        sufficiently accurate.
\end{enumerate}
During inference, we maintain only the sketches and lazily rehydrate
the exact K/V vectors when attention queries require high-fidelity
context.  Per-layer error budgets ensure that accumulated approximation
error does not degrade downstream predictions.  When the uncertainty
exceeds a threshold the system rehydrates and caches the exact K/V
vectors for that span.

\section{Prototype}
As a toy example, Algorithm~\ref{alg:kvsketch} illustrates
KV-sketching using singular value decomposition (SVD) to compress a
small matrix representing K/V states.  A practical system would
replace SVD with learnable projections and count-sketch hashing.

\begin{lstlisting}[language=Python, caption={Simple KV-sketching
prototype using SVD.}, label={alg:kvsketch}]
import numpy as np

def svd_sketch(matrix, rank=2):
    """Compress a matrix via truncated SVD and reconstruct."""
    u, s, vh = np.linalg.svd(matrix, full_matrices=False)
    u_r = u[:, :rank]
    s_r = s[:rank]
    vh_r = vh[:rank, :]
    # Sketch: low-rank factors (u_r, s_r, vh_r)
    return u_r, s_r, vh_r

def reconstruct(u_r, s_r, vh_r):
    """Rehydrate the approximate matrix from its sketch."""
    return (u_r @ np.diag(s_r)) @ vh_r

# Example K/V state: 10 tokens x 5 hidden dimension
state = np.random.rand(10, 5)
u_r, s_r, vh_r = svd_sketch(state, rank=2)
approx = reconstruct(u_r, s_r, vh_r)
error = np.linalg.norm(state - approx)
print(f"Approximation error: {error:.4f}")
\end{lstlisting}

The accompanying \texttt{code.py} file in this folder implements
this example and imports the \texttt{sciresearch\_ai} package to
demonstrate integration with the repository.

\section{Evaluation}
To evaluate KV-Sketching one can monitor:
\begin{itemize}
  \item \textbf{Speedup vs. quality:} compare throughput (tokens/s) and
        perplexity for the sketched model versus the baseline.
  \item \textbf{Memory savings:} measure the reduction in VRAM usage
        per one million tokens.
  \item \textbf{Latency:} compute p95 response times with and without
        rehydration.
\end{itemize}

\section{Failure modes}
Sketching introduces approximation error that can accumulate across
layers.  If the uncertainty predictor underestimates error, the model
may hallucinate or drop critical context.  Mitigation strategies
include conservative uncertainty thresholds and periodic rehydration
checkpoints.

\end{document}