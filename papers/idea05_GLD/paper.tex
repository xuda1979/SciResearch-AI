\documentclass{article}
\usepackage{amsmath}

\title{Global-then-Local Decoding (GLD)}
\author{Researcher}
\date{\today}

\begin{document}

\maketitle

\begin{abstract}
Global-then-Local Decoding decomposes generation into a planning stage
followed by constrained local decoding, reducing contradictions and
improving long-form coherence.
\end{abstract}

\section{Overview}
GLD first produces a high-level plan that outlines goals and
constraints. Local sections are then filled under this plan using
constrained decoding.

\section{Method}
The model initially emits a structured plan (e.g., JSON) specifying key
subgoals, evidence to include, and ordering. A checker validates the
plan before proceeding. During local decoding, the model is constrained
to adhere to the plan and include the specified evidence.

\section{Implementation}
The provided \texttt{code.py} creates a simple plan for a document and
fills sections with placeholder text, demonstrating how the
\texttt{sciresearch\_ai} module can track version information.

\section{Conclusion}
Separating planning from local generation offers a straightforward way
to improve coherence and enforce global constraints in long-form
outputs.

\end{document}
