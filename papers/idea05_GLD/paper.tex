\documentclass{article}
\usepackage{amsmath}

\title{Global-then-Local Decoding (GLD)}
\author{Researcher}
\date{\today}

\begin{document}

\maketitle

\section{Overview}
Global-then-Local Decoding (GLD) decomposes generation into two phases:
first a global plan is produced that outlines goals and constraints,
then local sections are filled under the plan using constrained decoding.
This approach reduces contradictions and improves coherence across long
documents.

\section{Method}
The model first emits a structured plan (e.g., in JSON) specifying
key subgoals, evidence to include, and ordering. A checker validates the
plan before proceeding. During local decoding, the model is constrained
to adhere to the plan and include the specified evidence.

\section{Implementation}
Our code example creates a simple plan for a document and then fills
sections using placeholder text. The \texttt{sciresearch\_ai} module is
imported for versioning information.

\end{document}