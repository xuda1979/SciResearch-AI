\documentclass{article}
\usepackage{amsmath}
\usepackage{hyperref}
\usepackage{listings}

\title{Idea 12: Differentially-Private Ephemeral Memory with Revocation}
\author{SciResearch-AI}
\date{\today}

\begin{document}
\maketitle

\begin{abstract}
Long-lived conversational agents risk accumulating sensitive or malicious
information in their internal caches.  Differentially-Private Ephemeral
Memory (DPEM) aims to enforce the ``right to be forgotten'' and limit
prompt‑injection persistence by associating each memory with a privacy budget
and a revocation token.  When the budget is exhausted or a valid
revocation request is received, the memory is purged from the model's
context.
\end{abstract}

\section{Problem}
Autoregressive LLMs accumulate context across multiple turns and may
implicitly memorize user inputs.  This persistent state makes it
difficult to honour deletion requests or to prevent toxic content from
resurfacing.  Na{"i}ve history truncation harms coherence, while full
retention violates privacy regulations.

\section{Method}
DPEM treats each stored fact or chunk as a tuple $(v,\varepsilon,\tau,
\mathit{revID})$ where $v$ is the value, $\varepsilon$ is a
differential privacy budget, $\tau$ is a time-to-live, and
$\mathit{revID}$ is a signed revocation token.  The agent's router
retrieves a memory only if its remaining budget exceeds a threshold
and its revocation token has not been submitted.  Each retrieval
decrements $\varepsilon$, modelling privacy loss.  A revocation
request carrying $\mathit{revID}$ immediately deletes the associated
memory.

\section{Prototype}
Listing~\ref{lst:dpem} shows a simple Python prototype of an
ephemerally private memory store.  Each entry decays over time and
can be revoked via a token.

\begin{lstlisting}[language=Python, caption={Prototype of a DPEM
store.}, label={lst:dpem}]
import time

class DPEMStore:
    def __init__(self):
        self.store = {}

    def add(self, key, value, epsilon=3, ttl=60, rev_id=None):
        # Store item with budget, expiry and revocation id
        self.store[key] = {
            "value": value,
            "epsilon": epsilon,
            "expiry": time.time() + ttl,
            "rev_id": rev_id,
        }

    def get(self, key):
        item = self.store.get(key)
        if not item:
            return None
        # Check expiry and budget
        if item["expiry"] < time.time() or item["epsilon"] <= 0:
            self.store.pop(key, None)
            return None
        # Decrement privacy budget on access
        item["epsilon"] -= 1
        return item["value"]

    def revoke(self, rev_id):
        # Remove all entries matching revocation id
        keys_to_delete = [k for k, v in self.store.items() if v.get("rev_id") == rev_id]
        for k in keys_to_delete:
            self.store.pop(k, None)

# Example usage
if __name__ == "__main__":
    mem = DPEMStore()
    mem.add("session_fact", "User’s secret", epsilon=2, ttl=5, rev_id="abc123")
    print(mem.get("session_fact"))  # retrieves and decrements epsilon
    time.sleep(6)
    print(mem.get("session_fact"))  # expired, returns None
    mem.add("another", "data", rev_id="xyz")
    mem.revoke("xyz")
    print(mem.get("another"))  # None after revocation
\end{lstlisting}

This prototype illustrates per-access budget decrementing and time-based
expiry. The accompanying \texttt{code.py} script imports the
\texttt{sciresearch\_ai} module to report the package version and
demonstrate integration with the repository. A production system
would integrate digital signatures for $\mathit{revID}$ and enforce
differential privacy accounting across multiple memories.

\section{Evaluation}
Useful metrics include:
\begin{itemize}
  \item \textbf{Forgetting latency:} time between a revocation request
        and the memory becoming inaccessible.
  \item \textbf{Leakage under membership tests:} probability that an
        attacker can infer whether a revoked fact was ever stored.
  \item \textbf{User trust:} satisfaction and compliance scores
        collected from participants issuing deletions.
\end{itemize}

\section{Failure modes}
Over-aggressive forgetting can degrade the user experience by causing
the agent to lose necessary context.  Under-estimating privacy loss
may leave sensitive information exposed.  Balancing
$\varepsilon$ and TTL per user or tenant is essential for safety and
utility.

\end{document}