\documentclass{article}
\usepackage{hyperref}
\title{Entropy-Thermostat Decoding (ETD)}
\author{SciResearch AI Team}
\date{\today}
\begin{document}
\maketitle
\begin{abstract}
Entropy-Thermostat Decoding (ETD) regulates the entropy of output tokens
to reduce verbosity and stabilise reasoning depth. It dynamically
adjusts the sampling temperature based on a target entropy schedule.
\end{abstract}
\section{Introduction}
Long-form answers can suffer from rambling and inconsistent reasoning
depths. ETD introduces a target entropy schedule over the answer; when
the model's output entropy deviates from this schedule, the decoding
temperature is adjusted to encourage more or less exploration.
\section{Method}
The algorithm monitors the entropy of the token probability
distribution during generation. If entropy spikes without new
subgoals, the temperature is lowered to reduce randomness. If entropy
is too low while the model is stuck, the temperature is increased. This
thermostat approach keeps the generation aligned with a desired entropy
profile.
\section{Implementation}
A prototype wraps the model's sampling loop. The wrapper computes the
entropy at each step and applies a function to adjust the temperature.
The schedule may be linear or piecewise, with optional checkpoints for
summarisation.
\section{Conclusion}
ETD provides a simple control mechanism to manage verbosity and
maintain focus during generation, improving clarity and efficiency.
\end{document}
