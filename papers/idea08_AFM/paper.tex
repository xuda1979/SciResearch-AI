\documentclass{article}
\usepackage{hyperref}
\title{Answer Futures Market (AFM)}
\author{SciResearch AI Team}
\date{\today}
\begin{document}
\maketitle
\begin{abstract}
The Answer Futures Market (AFM) is a novel mechanism for routing computational budget in a mixture-of-experts or agents system. Specialists bid their confidence and associated cost to answer a question; a router allocates resources to those with the highest expected value of improvement.
\end{abstract}
\section{Introduction}
Language models can vary widely in capability, cost, and domain expertise. AFM formalizes a market where each model or head emits a probability of correctness and a bid reflecting its cost. A router chooses which models to consult based on expected utility.
\section{Method}
In AFM, each specialist head outputs a tuple $(p_{\text{correct}}, \sigma^2, \text{cost})$. Using proper scoring rules, the router determines which bids offer the highest expected value. The selection can be adaptive: initial cheap models run first, and more expensive ones are invoked only if uncertainty remains high.
\section{Implementation}
A simple prototype can use two models: a small model and a large model. Each returns a confidence score. The router calls the large model only when the small model's confidence is below a threshold. This mechanism can extend to multiple experts with dynamic bidding.
\section{Conclusion}
Answer Futures Market offers a structured way to manage computational resources while improving answer quality by leveraging multiple specialized models.
\end{document}
