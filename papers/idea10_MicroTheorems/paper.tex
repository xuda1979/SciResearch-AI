\documentclass{article}
\title{Knowledge Fossilization via Micro-Theorems}
\author{Research Team}
\date{\today}
\begin{document}
\maketitle

\begin{abstract}
Micro-theorems encode domain facts as small verifiable units. By
embedding tests or proofs with each theorem, models can preserve
critical knowledge across updates and avoid regressions.
\end{abstract}

\section{Introduction}
Large language models are prone to forgetting or misrepresenting
established facts when updated or fine-tuned. To counteract this drift,
we propose knowledge fossilisation via micro-theorems. Each important
domain fact is formalised as a small theorem with an accompanying test
or proof. During generation, these micro-theorems can be referenced or
imported, ensuring that critical relationships remain consistent across
model updates.

\section{Method}
A micro-theorem captures a factual relationship (e.g., ``ICD--10 code X
corresponds to condition Y'') and includes a short proof or
verification function. By organising the knowledge base into such
units, we can enforce that any output using these facts must pass the
associated tests. Micro-theorems can be versioned and expired when the
underlying fact changes.

\section{Implementation}
To implement this idea, we build a library of micro-theorems as Python
modules with unit tests. The language model is restricted to import from
this library when generating answers in regulated domains. A
constrained decoding mechanism ensures that references to domain
knowledge call the corresponding micro-theorem functions. If a
generated statement fails its test, the model revises the statement or
abstains.

\section{Conclusion}
Organising factual relationships into micro-theorems supplies a durable
scaffold that constrains model outputs and simplifies auditing for
drift.

\end{document}
