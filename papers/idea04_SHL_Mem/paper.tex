\documentclass{article}
\usepackage{amsmath,amssymb}

\title{Semantic-Hash Lattice Memory (SHL-Mem)}
\author{Researcher}
\date{\today}

\begin{document}

\maketitle

\section{Introduction}
Semantic-Hash Lattice Memory (SHL-Mem) is a proposed long-context memory
architecture that uses multiple locality-sensitive semantic hashes to organize
and retrieve information from very long sequences. By mapping spans of the
input into hash buckets along several semantic dimensions (e.g., entities,
relations, time), SHL-Mem allows efficient retrieval via lattice joins.

\section{Method}
Each span of tokens is hashed according to several functions corresponding
to different semantic views. The resulting hash codes form a lattice of sets.
Queries traverse the lattice by intersecting hash buckets that share
similar semantic content, providing a small candidate set for attention.

\section{Implementation}
In our demonstration code, we build a simplified version of SHL-Mem.
We generate random hash codes for spans and compute intersections to
approximate retrieval. The \texttt{sciresearch\_ai} module is imported
for versioning.

\end{document}
