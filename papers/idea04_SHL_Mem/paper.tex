\documentclass{article}
\usepackage{amsmath,amssymb}

\title{Semantic-Hash Lattice Memory (SHL-Mem)}
\author{Researcher}
\date{\today}

\begin{document}

\maketitle

\begin{abstract}
Semantic-Hash Lattice Memory (SHL-Mem) organizes extremely long
contexts using multiple locality-sensitive semantic hashes. By mapping
spans into hash buckets across several semantic dimensions (e.g.,
entities, relations, time), SHL-Mem retrieves relevant information via
lattice joins.
\end{abstract}

\section{Introduction}
SHL-Mem is a long-context memory architecture that leverages multiple
semantic hash functions to organize and retrieve information from very
long sequences. Mapping input spans into hash buckets along semantic
dimensions enables efficient retrieval through lattice joins.

\section{Method}
Each span of tokens is hashed according to several functions
corresponding to different semantic views. The resulting hash codes
form a lattice of sets. Queries traverse the lattice by intersecting
hash buckets that share similar content, yielding a small candidate set
for attention.

\section{Implementation}
The demonstration code generates random hash codes for spans and
computes intersections to approximate retrieval. The
\texttt{sciresearch\_ai} module is imported for versioning.

\section{Conclusion}
SHL-Mem shows how structured hashing can support scalable retrieval in
ultra-long sequences while keeping lookup costs manageable.

\end{document}
