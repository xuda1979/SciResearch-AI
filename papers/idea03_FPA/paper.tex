\documentclass{article}
\usepackage[utf8]{inputenc}
\title{Field-Programmable Attention for Efficient Long Contexts}
\author{Research Team}
\date{\today}

\begin{document}
\maketitle

\begin{abstract}
This paper introduces Field-Programmable Attention (FPA), a technique to
program sparse attention patterns based on task-specific signals to improve
long-context utilization and reduce computational cost.
\end{abstract}

\section{Introduction}
Large language models struggle to maintain focus over long contexts, often
losing critical information. Dense attention is expensive and may ignore
important sections.

\section{Method}
FPA pre-compiles a block-sparse attention mask using cues such as titles,
headings, anchors, and entities. The mask generator outputs patterns that
inform the attention mechanism to attend selectively.

\section{Implementation}
The implementation uses the \texttt{sciresearch\_ai} module. The accompanying
\texttt{code.py} demonstrates how to create a simple attention mask for a toy
sequence.

\section{Peer Review}
Field ‑ Programmable Attention proposes compiling task ‑ specific sparse
attention patterns ahead of time.  This design could reduce compute
requirements for long contexts and focus the model on relevant spans.
However, pre ‑ defining attention masks assumes that salient regions can be
identified reliably from high ‑ level cues such as titles or headings.  In
practice, the importance of a section may depend on the question, so static
masks could miss critical information.  It would be valuable to evaluate
adaptive schemes where the mask generator is conditioned on the query.
Additionally, the overhead of generating and storing masks for large
documents needs to be considered.  The current prototype operates on toy
sequences; future work should benchmark FPA on realistic workloads and
compare its performance and memory footprint against state ‑ of ‑ the ‑ art
sparse attention algorithms.

\end{document}
