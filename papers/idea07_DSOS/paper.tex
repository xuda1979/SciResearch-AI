\documentclass{article}
\usepackage{hyperref}
\title{Deterministic Shadow-OS for Agents (DSOS)}
\author{SciResearch AI Team}
\date{\today}
\begin{document}
\maketitle
\begin{abstract}
Deterministic Shadow-OS for Agents (DSOS) makes AI agents safer when
interacting with operating systems and web interfaces. DSOS forces the
agent to produce a deterministic intention log of actions to be
executed in a sandbox, enabling human review and rollback before
committing changes.
\end{abstract}
\section{Introduction}
Tool-using agents often struggle with unpredictable UI elements and
side effects. DSOS introduces a shadow environment that records planned
filesystem diffs, HTTP actions, and UI selectors. Actions are first
replayed deterministically in this environment and only committed upon
review.
\section{Method}
DSOS consists of three steps: (1) the agent writes a plan of actions
with semantic selectors; (2) the plan is executed in a deterministic,
headless shadow OS or browser, producing a diff and logs; (3) a
verifier or human approves the diff before real execution. This ensures
actions are reversible and auditable.
\section{Implementation}
A minimal prototype uses a headless browser and file system to simulate
actions. The agent's plan can be represented as JSON. After simulation,
the diff is presented for approval. Only upon approval does the agent
execute the actions on the real system.
\section{Conclusion}
DSOS improves agent reliability by separating planning, verification,
and execution stages, reducing unintended side effects and allowing
human oversight.
\end{document}
