\RequirePackage{filecontents}
\begin{filecontents*}{refs.bib}
@article{lu2024aiscientist,
  title={The AI Scientist: Towards Fully Automated Open-Ended Scientific Discovery},
  author={Lu, Chris and Lu, Cong and Lange, Robert Tjarko and Foerster, Jakob and Clune, Jeff and Ha, David},
  year={2024},
  eprint={2408.06292},
  archivePrefix={arXiv},
  primaryClass={cs.LG}
}
@inproceedings{yao2023react,
  title={ReAct: Synergizing Reasoning and Acting in Language Models},
  author={Yao, Shunyu and Zhao, Jeffrey and Yu, Dian and Du, Nan and Shafran, Izhak and Narasimhan, Karthik and Cao, Yuan},
  booktitle={ICLR},
  year={2023}
}
@article{schick2023toolformer,
  title={Toolformer: Language Models Can Teach Themselves to Use Tools},
  author={Schick, Timo and Dwivedi-Yu, Jane and Dess{\`i}, Roberto and Raileanu, Roberta and Lomeli, Maria and Hambro, Eric and Zettlemoyer, Luke and Cancedda, Nicola and Scialom, Thomas},
  year={2023},
  eprint={2302.04761},
  archivePrefix={arXiv}
}
@article{wang2022selfconsistency,
  title={Self-Consistency Improves Chain of Thought Reasoning in Language Models},
  author={Wang, Xuezhi and Wei, Jason and Schuurmans, Dale and Le, Quoc and Chi, Ed H. and Narang, Sharan and Zhou, Denny},
  year={2022},
  eprint={2203.11171},
  archivePrefix={arXiv}
}
@article{yao2023tot,
  title={Tree of Thoughts: Deliberate Problem Solving with Large Language Models},
  author={Yao, Shunyu and Yu, Dian and Zhao, Jeffrey and Shafran, Izhak and Griffiths, Thomas L. and Cao, Yuan and Narasimhan, Karthik},
  year={2023},
  eprint={2305.10601},
  archivePrefix={arXiv}
}
@article{wang2023voyager,
  title={Voyager: An Open-Ended Embodied Agent with Large Language Models},
  author={Wang, Guanzhi and Xie, Yuqi and Jiang, Yunfan and Mandlekar, Ajay and Xiao, Chaowei and Zhu, Yuke and Fan, Linxi and Anandkumar, Anima},
  year={2023},
  eprint={2305.16291},
  archivePrefix={arXiv}
}
@online{k8sAutoscaler,
  title={Node Autoscaling (Cluster Autoscaler)},
  year={2024},
  url={https://kubernetes.io/docs/concepts/cluster-administration/node-autoscaling/}
}
@online{kueueDocs,
  title={Kueue: Kubernetes-native Job Queueing},
  year={2025},
  url={https://kueue.sigs.k8s.io/}
}
@online{argoDocs,
  title={Argo Workflows: The Workflow Engine for Kubernetes},
  year={2025},
  url={https://argo-workflows.readthedocs.io/en/latest/}
}
@online{mlflowTracking,
  title={MLflow Tracking},
  year={2025},
  url={https://mlflow.org/docs/latest/ml/tracking/}
}
@online{minioS3,
  title={MinIO: S3 API Compatibility},
  year={2025},
  url={https://docs.min.io/community/minio-object-store/reference/s3-api-compatibility.html}
}
@online{semanticscholarAPI,
  title={Semantic Scholar Academic Graph API},
  year={2025},
  url={https://api.semanticscholar.org/api-docs}
}
@online{crossrefAPI,
  title={Crossref REST API},
  year={2025},
  url={https://www.crossref.org/documentation/retrieve-metadata/rest-api/}
}
@online{arxivAPI,
  title={arXiv API Access},
  year={2025},
  url={https://info.arxiv.org/help/api/index.html}
}
@online{podSecurityStandards,
  title={Kubernetes Pod Security Standards},
  year={2025},
  url={https://kubernetes.io/docs/concepts/security/pod-security-standards/}
}
@online{networkPolicies,
  title={Kubernetes Network Policies},
  year={2024},
  url={https://kubernetes.io/docs/concepts/services-networking/network-policies/}
}
@online{gvisor,
  title={gVisor: Application Kernel for Containers},
  year={2025},
  url={https://github.com/google/gvisor}
}
@online{sigstore,
  title={Sigstore Cosign: Container signing and verification},
  year={2025},
  url={https://docs.sigstore.dev/cosign/overview/}
}
@online{slsa,
  title={SLSA: Supply-chain Levels for Software Artifacts},
  year={2025},
  url={https://slsa.dev/}
}
\end{filecontents*}

\documentclass[11pt]{article}
\usepackage[a4paper,margin=1in]{geometry}
\usepackage{hyperref}
\usepackage{graphicx}
\usepackage{tikz}
\usetikzlibrary{arrows.meta,positioning,fit,calc,shadows.blur,decorations.pathreplacing,shapes.multipart}
\usepackage{listings}
\usepackage{xcolor}
\usepackage{booktabs}
\usepackage{enumitem}
\usepackage[numbers,sort&compress]{natbib}
\usepackage{titlesec}

\titleformat{\section}{\large\bfseries}{\thesection}{1em}{}
\titleformat{\subsection}{\normalsize\bfseries}{\thesubsection}{1em}{}

\lstdefinestyle{code}{
  basicstyle=\ttfamily\small,
  breaklines=true,
  frame=single,
  columns=fullflexible,
  keepspaces=true,
  showstringspaces=false
}

\newcommand{\compbox}[3]{%
  \begin{tikzpicture}[>=Latex, node distance=6mm]
    \node[draw, rounded corners, fill=gray!5, inner sep=4mm, minimum width=#1, align=left, blur shadow] (box) {#3};
    \node[anchor=north west, font=\bfseries] at ($(box.north west)+(2mm,-2mm)$) {#2};
  \end{tikzpicture}
}

\begin{document}
\begin{center}
{\LARGE \textbf{Engineering Design Spec (MVP): SciResearch-AI Autonomous Research Paper System}}\\[6pt]
\textbf{Scope:} End-to-end automated scientific papers with code execution, results capture, and LaTeX build.\\[2pt]
\textbf{Non-goals (MVP):} Physical experiments; complex multi-tenant event buses; advanced supply-chain hardening.\\[4pt]
\textit{This MVP focuses on essential capabilities for reliability, reproducibility, and speed. Phase 2 lists optional extensions (event bus, signed images, SLSA, deep reviewer workflows).}
\end{center}

\section{Executive Summary}
Agentic planning (ReAct, Self-Consistency, ToT \cite{yao2023react,wang2022selfconsistency,yao2023tot}) drives autonomous research \cite{lu2024aiscientist}. Core pipeline: plan \textrightarrow{} synthesize code/tests \textrightarrow{} run jobs \textrightarrow{} analyze \textrightarrow{} build paper with numeric assertions. Infra (MVP): K8s Jobs (optional), Argo + Kueue for orchestration, MLflow + MinIO for tracking/artifacts.

\section{MVP Scope \& Deliverables}
\begin{itemize}[leftmargin=1.4em]
  \item Planner: literature retrieval (arXiv/Crossref), idea generation, normalized research\_spec.yaml
  \item Synthesizer: code + tests, simple container build, Argo DAG for variants; basic TTC knobs (k, b, L)
  \item Execution: K8s Jobs (single cluster) or local runner fallback; queues: cpu, gpu-spot, gpu-high (configurable)
  \item Analysis: MLflow metrics, artifact upload to S3/MinIO, canonical results.json
  \item Paper Builder: LaTeX templating, citation check, \textbf{numeric assertion} check vs results.json
  \item Reviewer: simple score with iterate-or-accept thresholds (T, N\_max)
  \item Security (baseline): non-root images, minimal NetPol, secret mounts, egress allow-list to storage/MLflow
  \item CI: lint+unit, build, smoke E2E on CPU; optional sign/verify in Phase 2
\end{itemize}

\section{MVP Roadmap \& Acceptance Criteria}
\subsection*{Phased Implementation (2–3 weeks)}
\begin{enumerate}[leftmargin=1.4em]
  \item \textbf{Week 1:}\ \emph{Core pipeline}\; Planner \Rightarrow Synthesizer \Rightarrow Local/K8s Runner \Rightarrow Analysis \Rightarrow Paper Builder
    \begin{itemize}
      \item Parse literature and draft \texttt{research\_spec.yaml}; create minimal bib (arXiv/Crossref)
      \item Generate Python code + tests; run locally; log to MLflow; push artifacts to MinIO
      \item Build LaTeX with numeric assertion check vs \texttt{results.json}
    \end{itemize}
  \item \textbf{Week 2:}\ \emph{K8s orchestration + queues}\; Argo DAG + Kueue queues (cpu/gpu-spot/gpu-high), TTC(k)
    \begin{itemize}
      \item Submit variants via Argo; Kueue admission; basic autoscaling policy
      \item Reviewer loop with thresholds (T, N\_max); iterate or accept
    \end{itemize}
  \item \textbf{Week 3:}\ \emph{Hardening \/ Observability}
    \begin{itemize}
      \item Baseline security: non-root images, minimal NetworkPolicies, secrets from KMS/Vault
      \item Dashboards for GPU util, queue depth, run success, cost projections
      \item CI: lint+unit, build, CPU smoke E2E; rollback on failures
    \end{itemize}
\end{enumerate}
\subsection*{Acceptance Criteria (MVP)}
\begin{itemize}[leftmargin=1.4em]
  \item End-to-end run produces a LaTeX paper whose numeric claims match \texttt{results.json}
  \item At least one GPU and one CPU variant can be executed via Argo+Kueue (or local fallback)
  \item Reproducibility: image digest, git SHA, object version IDs recorded in MLflow and \texttt{results.json}
  \item TTC: self-consistency parameter \(k\) reduces cost to meet budget without failing the run
\end{itemize}

\section{Minimal Local Development \& Runbook (Windows/conda)}
\begin{lstlisting}[style=code,language=bash,caption={Local smoke run (CPU)}]
# create and activate env
conda create -n sciresearch-ai python=3.11 -y
conda activate sciresearch-ai
pip install -r requirements.txt

# run minimal pipeline
echo "{\"title\":\"demo\",\"budget_usd\":10}" > projects/demo/spec.json
python generate_paper.py
# outputs under projects/<name>/ paper/*.tex and artifacts in MinIO/MLflow (if configured)
\end{lstlisting}
\noindent\textit{Optional:} configure MLflow and MinIO endpoints via environment variables for artifact tracking.

\section{Interfaces, Schemas, and Contracts}
\subsection{Canonical Artifacts}
\begin{lstlisting}[style=code,language=json,caption={results.json JSON Schema (excerpt)}]
{
  "$schema":"https://json-schema.org/draft/2020-12/schema",
  "title":"Results",
  "type":"object",
  "required":["experiment_id","variants","figures","artifacts"],
  "properties":{
    "experiment_id":{"type":"string", "pattern":"^exp_[0-9_]+$"},
    "seed_policy":{"type":"object","properties":{"master_seed":{"type":"integer"}}},
    "variants":{
      "type":"object",
      "additionalProperties":{
        "type":"object",
        "required":["rmse","runtime_sec"],
        "properties":{
          "rmse":{"type":"number"},
          "runtime_sec":{"type":"number"},
          "seeds":{"type":"array","items":{"type":"integer"}}
        }
      }
    },
    "mlflow_run_urls":{"type":"array","items":{"type":"string","format":"uri"}},
    "figures":{"type":"array","items":{"type":"string"}},
    "artifacts":{"type":"array","items":{"type":"string"}}
  }
}
\end{lstlisting}

\begin{lstlisting}[style=code,language=yaml,caption={research_spec.yaml (normalized fields)}]
version: 1
id: rs_2025_08_21_0001
title: "..."
budget_usd: 100
deadline_hours: 24
seed_policy: { master_seed: 172421 }
datasets:
  - name: synthetic_regression_v1
methods:
  - baseline_greedy
  - self_consistency_k=10
metrics: [rmse, runtime_sec, gpu_hours]
\end{lstlisting}

\subsection{Event Bus (Kafka/NATS) Payloads w/ Versioning}
\textit{Optional (Phase 2). Not required for MVP; direct component invocation and Argo artifacts suffice.}
\begin{lstlisting}[style=code,language=json,caption={Event: research.spec.created v1}]
{
  "version":"1.0",
  "spec_url":"s3://bucket/specs/rs_2025_08_21_0001.yaml",
  "spec_hash":"sha256:...",
  "trace_id":"trc-abc123",
  "created_at":"2025-08-21T12:00:00Z"
}
\end{lstlisting}

\begin{lstlisting}[style=code,language=json,caption={Event: research.results.available v1}]
{
  "version":"1.0",
  "results_url":"s3://bucket/results/exp_2025_08_21_1234/results.json",
  "experiment_id":"exp_2025_08_21_1234",
  "mlflow_experiment":"exp_42",
  "digest":"sha256:...",
  "trace_id":"trc-abc123"
}
\end{lstlisting}

\subsection{AuthN/Z}
All internal HTTP/gRPC endpoints require **mTLS** (service accounts) and a short‑lived **JWT** with audience scoping. Topic ACLs restrict publish/subscribe per component.

\section{Component Blueprints (Updated Diagrams)}
\subsection{A. Research Planner}
\begin{figure}[h]
\centering
\compbox{12cm}{A.1 Internals}{
\begin{minipage}{0.95\linewidth}
\textbf{Submodules:} Ingest, Literature (S2/arXiv/Crossref APIs \cite{semanticscholarAPI,arxivAPI,crossrefAPI}), Idea Gen (GPT‑5), Novelty Check (Vector DB), Spec Writer (YAML).\\
\textbf{Hardening:} egress allow‑list to literature hosts; cache PDFs; rate‑limit with backoff.\\
\textbf{Outputs:} \texttt{research\_spec.yaml}, \texttt{refs.json}, cached PDFs, bib keys.
\end{minipage}
}
\caption{Planner internals with controlled egress and caching.}
\end{figure}

\subsection{B. Code Synthesizer \& Orchestrator}
\begin{figure}[h]
\centering
\compbox{12cm}{B.1 Internals}{
\begin{minipage}{0.95\linewidth}
ReAct tool-calls \cite{yao2023react} and optional Toolformer-style tool learning \cite{schick2023toolformer}. Generates code + tests, builds image, signs with cosign \cite{sigstore}, emits Argo DAG. TTC knobs embedded as parameters.
\end{minipage}
}
\caption{Synthesizer with signed images and TTC knobs.}
\end{figure}

\subsection{C. Execution Sandbox (K8s Jobs)}
\begin{figure}[h]
\centering
\compbox{12cm}{C.1 Internals}{
\begin{minipage}{0.95\linewidth}
Kueue queues map variants to node pools; gVisor runtime; PSS restricted; NetworkPolicies deny-all + allow S3/MLflow only. Images verified (cosign), SBOM checked (SLSA) \cite{sigstore,slsa}.
\end{minipage}
}
\caption{Execution layer security and queueing.}
\end{figure}

\subsection{D. Analysis \& Visualization}
\begin{figure}[h]
\centering
\compbox{12cm}{D.1 Internals}{
\begin{minipage}{0.95\linewidth}
Aggregates MLflow metrics; computes stats; generates figs/tables; writes canonical \texttt{results.json}; provenance links to image digests and git SHAs.
\end{minipage}
}
\caption{Analysis with provenance \& lineage.}
\end{figure}

\subsection{E. Paper Builder (Validation Added)}
\begin{figure}[h]
\centering
\compbox{12cm}{E.1 Internals}{
\begin{minipage}{0.95\linewidth}
Template filler + writer (GPT‑5), \textbf{Assertion Checker}: fails build if any number cited in text mismatches \texttt{results.json}; DOI lookup; deterministic \texttt{latexmk}.
\end{minipage}
}
\caption{Paper generation with numeric assertion checks.}
\end{figure}

\subsection{F. Reviewer \& Iteration Manager (State Machine)}
\begin{figure}[h]
\centering
\begin{tikzpicture}[>=Latex, node distance=12mm, font=\small]
\tikzstyle{st}=[draw, rounded corners, fill=white, inner sep=2mm, minimum width=3.2cm, align=center]
\node[st] (draft) {DRAFT};
\node[st, right=2.8cm of draft] (review) {REVIEW};
\node[st, right=2.8cm of review] (improve) {IMPROVE};
\node[st, below=12mm of review] (accept) {ACCEPT};
\node[st, below=12mm of improve] (abandon) {ABANDON};
\draw[->] (draft) -- node[above]{submit} (review);
\draw[->] (review) -- node[above]{score < T \& actionable} (improve);
\draw[->] (review) -- node[left]{score ≥ T \& coverage OK} (accept);
\draw[->] (improve) -- node[right]{N tries ≥ Nmax or budget hit} (abandon);
\draw[->] (improve) -- node[above]{new exps / edits} (review);
\end{tikzpicture}
\caption{Iteration state machine with thresholds \(T\), \(N_{\max}\) and budget stops.}
\end{figure}

\section{Budgeted Test-Time Compute (TTC) Controller}
\subsection{Knobs and Mapping}
\textbf{MVP simplification:} use only self-consistency samples \(k\). Beam width \(b\) and depth \(L\) for structured search (e.g., Tree of Thoughts) are \textit{optional Phase 2} knobs.
\[\mathrm{gpu\_hours} \approx \sum_{i} \frac{k_i \cdot c_i}{\eta_{util}}\]
where \(c_i\) is per-sample cost estimated from prior runs; \(\eta_{\text{util}}\) is utilization. Controller chooses \(k\) to satisfy:
\[
\text{gpu\_hours} \cdot p_{\text{gpu}} + \text{cpu\_hours} \cdot p_{\text{cpu}} \le \text{budget\_usd}.
\]
On shortfall, reduce \(k\) (only, in priority order), or move jobs from \texttt{gpu-high} to \texttt{gpu-spot} with preemption tolerance.

\subsection{Policy (executable sketch)}
\begin{lstlisting}[style=code,caption={Budget controller pseudocode (MVP k-only)}]
def allocate_ttc(spec, priors, prices):
    plan = []
    for v in spec.methods:
        est_gpu = priors.estimate_gpu_hours(k=v.k)
        plan.append((v, est_gpu))
    while cost(plan, prices) > spec.budget_usd:
        for v,_ in sorted(plan, key=lambda x: x[1], reverse=True):
            if v.k > 1: v.k -= 1; break
        else:
            raise BudgetError("Cannot fit budget")
    return plan
\end{lstlisting}

\subsection{Advanced Inference-Time Algorithms}
Beyond best-of-\(k\) voting, the controller can escalate to more structured search when additional budget is available:
\begin{itemize}[leftmargin=1.4em]
\item \textbf{Tree of Thoughts (ToT):} explore a reasoning tree with beam width \(b\) and depth \(L\), pruning by heuristic or learned scores \cite{yao2023tot}.
\item \textbf{Multi-agent debate:} spawn parallel agents and select a consensus answer after critique rounds to improve factuality \cite{yao2023react}.
\item \textbf{Reflection loops:} critique and revise drafts until a checklist passes, reducing formatting and citation errors.
\end{itemize}
The controller allocates budget across these techniques and falls back to self-consistency for low budgets.

\section{Scheduling and Queues}
\begin{itemize}[leftmargin=1.4em]
\item \textbf{Queues:} \texttt{cpu}, \texttt{ingest}, \texttt{gpu-spot}, \texttt{gpu-high}. Kueue admits/suspends; Argo parallelism bounds TTC breadth \cite{kueueDocs,argoDocs}.
\item \textbf{Autoscaling:} Cluster Autoscaler grows node pools on pending; max nodes per pool pinned to budget \cite{k8sAutoscaler}.
\end{itemize}

\section{Security, Secrets, and Supply Chain}
\begin{itemize}[leftmargin=1.4em]
\item \textbf{Baseline (MVP):} non-root containers; minimal NetworkPolicies (deny-all, allow S3/MLflow/Postgres); secret mounts via KMS/Vault; short-lived tokens; optional mTLS in single cluster.
\item \textbf{Egress:} allow-list only to storage/MLflow; literature retrievers limited to specific hosts.
\item \textbf{Phase 2 (Optional):} image signing (cosign), SBOM + SLSA provenance, admission verify.
\end{itemize}

\section{Reproducibility \& Lineage}
\begin{itemize}[leftmargin=1.4em]
\item Pin images by \textbf{digest}, code by \textbf{git SHA}, data by \textbf{object version ID}.
\item \textbf{Seeds:} global master seed with deterministic sub‑seeds per variant; recorded in \texttt{results.json}.
\item MLflow run tags: \texttt{image\_digest}, \texttt{git\_sha}, \texttt{spec\_id}, \texttt{seed}.
\end{itemize}

\section{Paper Validation (Hard Checks)}
The Paper Builder fails compile if:
\begin{enumerate}[leftmargin=1.4em]
\item Any numeric literal referenced with tag \verb|\result{path.to.metric}| does not equal the value in \texttt{results.json}.
\item Any figure path in LaTeX not present in S3.
\item Any citation key missing from \texttt{refs.bib}.
\end{enumerate}

\noindent\textbf{Usage in LaTeX:}
\begin{lstlisting}[style=code,language=TeX,caption={Numeric reference macro}]
% \result{variants.self_consistency_k10.rmse} -> 0.191
\newcommand{\result}[1]{\input{.cache/results/#1.tex}}
\end{lstlisting}

\section{Observability \& SLOs}
\begin{itemize}[leftmargin=1.4em]
\item \textbf{SLOs:} P95 time to first experiment $<$ 20 min; P95 paper build success $>$ 99\%; Reviewer loop $\leq 3$ iterations in 90\% of runs.
\item \textbf{Alerts:} queue wait $>$ 15m (warn), 30m (crit); OOMKilled count $>$ 3/run; cost projection exceed budget.
\item \textbf{Dashboards:} GPU util, queue depth, success rates, cost per run, reviewer scores over time.
\end{itemize}

\section{CI/CD and Environments}
\begin{itemize}[leftmargin=1.4em]
\item \textbf{Stages (MVP):} dev (cpu only) $\rightarrow$ staging (gpu-spot) $\rightarrow$ prod (gpu-high).
\item \textbf{Pipeline:} lint+unit $\rightarrow$ build $\rightarrow$ smoke run (baseline) $\rightarrow$ promote. \textit{Optional:} SBOM+scan, sign+verify in Phase 2.
\item \textbf{Rollback:} revert to previous image on elevated failure rate.
\end{itemize}

\section{Testing Strategy}
\begin{itemize}[leftmargin=1.4em]
  \item \textbf{Unit (MVP):} schema validation; TTC controller (k-only); numeric assertion macro
  \item \textbf{Integration:} local run produces \texttt{results.json}; Paper build fails on mismatched numbers
  \item \textbf{E2E Smoke:} CPU-only pipeline under 5 minutes; accepts or iterates within N\_max
\end{itemize}

\section{Failure Modes and Automated Remediation}
\begin{tabular}{@{}p{3.7cm}p{6.8cm}p{4.6cm}@{}}
\toprule
Failure & Detection & Automated Remediation \\\midrule
Queue starvation & wait\_time $>$ 30m & Reduce $k$ by 50\%, move to gpu-spot; if still pending, downgrade to cpu fallback \\
Compile error & latexmk nonzero & Re-generate missing refs, re-fetch figs; on 2nd fail, open ticket and attach logs \\
Novelty collision & high sim score & Planner rewrites hypothesis or picks next idea; add filter term to queries \\
Budget exceed & cost projection $>$ budget & TTC controller shrinks $k\rightarrow b\rightarrow L$; cancel low-ROI variants \\
\bottomrule
\end{tabular}

\section{Appendix: Example Manifests}
\subsection*{Kueue/Argo (GPU job with TTC parallelism)}
\begin{lstlisting}[style=code,language=yaml]
apiVersion: argoproj.io/v1alpha1
kind: Workflow
metadata: { generateName: research-run- }
spec:
  entrypoint: main
  templates:
  - name: main
    dag:
      tasks:
      - name: build
        template: docker-build
      - name: k10
        dependencies: [build]
        template: kueue-job
        arguments: { parameters: [{name: k, value: 10}] }
  - name: kueue-job
    inputs:
      parameters:
      - name: k
    resource:
      action: create
      manifest: |
        apiVersion: batch/v1
        kind: Job
        metadata: { name: train-k{{inputs.parameters.k}} }
        spec:
          parallelism: {{inputs.parameters.k}}
          template:
            spec:
              runtimeClassName: gvisor
              containers:
              - name: runner
                image: registry/research@sha256:...
                resources:
                  requests: { nvidia.com/gpu: 1, cpu: "4", memory: "16Gi" }
                  limits:   { nvidia.com/gpu: 1, cpu: "4", memory: "16Gi" }
                env:
                  - { name: MLFLOW_TRACKING_URI, value: "http://mlflow:5000" }
                  - { name: S3_ENDPOINT, value: "http://minio:9000" }
              restartPolicy: Never
\end{lstlisting}

\bibliographystyle{plainnat}
\bibliography{refs}
\end{document}
